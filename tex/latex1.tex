\documentclass{letter}
\usepackage[utf8]{inputenc}

\usepackage[T1]{fontenc}

\usepackage{graphicx}

%opening
\title{LaTeX c'est word pour les barbus}
\author{Vincent Loy}

\begin{document}



\includegraphics[height=45pt, width=100pt]{gmp.png}  

Bonjour, je lis actuellement un cours traitant de LaTeX !

LaTex ; une techno du \textsc{XX}\ième siècle !

Lorem ipsum dolor sit amet, consectetur adipiscing elit. Curabitur mattis auctor massa. Pellentesque tincidun 
lacus vel malesuada ullamcorper, ipsum nibh interdum odio, eu malesuada nulla nisi a ex. Quisque tempor hendr 
Praesent cursus dapibus augue, eu eleifend magna sodales sed. Curabitur dapibus auctor ante, at sagittis lacu. 
Pellentesque habitant morbi tristique senectus et netus et malesuada fames ac turpis egestas. Quisque euismod 
sollicitudin eros elementum in. Integer gravida, eros a eleifend finibus, dui est euismod massa, sit amet cur
Lorem ipsum dolor sit amet, consectetur adipiscing elit. Sed sollicitudin nulla libero, nec dictum neque posu. 
Cras finibus elit vel ligula vestibulum, at volutpat dolor porta. Nullam eget mi hendrerit sem vehicula iacul.

\textit{Lorem ipsum dolor sit amet, \emph{consectetuer} adipiscing elit.}
Lorem ipsum dolor sit amet, \emph{consectetuer} adipiscing elit.


Gros paragraphe.

% une liste !
\begin{itemize}

\item Un canard.
\item Un mammouth.
\item Un canard.
\item Un mammouth.
\item Un canard.
\item Un mammouth.
\item[@] Une pintade. % En plaçant un @ entre crochets après \item, j'ai transformé la puce en @
\item[0] Un lapin.

\end{itemize}

{\fontfamily{pag}\selectfont mon bout de texte}


\begin{description}
\item[Un canard :] bestiole qui fait coin.
\item[Un poulpe :] bestiole qui fait bloub.
\item[Un ornithorynque :] bestiole qui fait rire.
\item[Un ours :] bestiole qui fait mal.
\end{description}

\end{document}