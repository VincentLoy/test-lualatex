\documentclass{article}
\usepackage{luacode}
 
\begin{document}

\begin{luacode}
  function readAll(file)
    local f = io.open(file, "rb")
    local content = f:read("*all")
    f:close()
    return content
  end

 local json = require ("lua/dkjson")
 local tbl = {
    animals = { "dog", "cat", "aardvark" },
    instruments = { "violin", "trombone", "theremin" },
    bugs = json.null,
    trees = nil
  }

local str = json.encode (tbl.animals[1], { indent = true })

tex.sprint(str)
\end{luacode}

% une liste !
  Un liste de nombres aléatoires :
  \begin{itemize}
    \item \directlua{ tex.print(math.random()) }
    \item \directlua{ tex.print(math.random()) }
    \item \directlua{ tex.print(math.random()) }
    \item \directlua{ tex.print(math.random()) }
    \item \directlua{ tex.print(math.random()) }
  \end{itemize}

\end{document}